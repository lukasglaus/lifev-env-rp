\chapter{Program Development Conventions} 

\section{Documentation Rules} 


All classes and methods should have
documentations lines which explain in a concise but thorough way it
usage. Documentation should be provided  with methods
\emph{declarations}

The amount of information contained in the documentation of functions
and methods should augment in the following order (unless there are
special needs)
\begin{itemize}
\item Inlined private methods, constructors and destructor;
\item Inlined methods;
\item Public methods;
\item User interface methods.
\end{itemize}
By \emph{user interface methods} we intend those methods that are
meant to be directly called by the user of the library. Those methods
should be the best documented.

\subsection{Methods and function documentation}
\label{sec:fdoc}
I present a template for documentation of a method, which is meant to
serve as example.


\begin{verbatim}

void Euler::Gudonov(Real nflux[4], Real const lsol[4],
                    Real const rsol[4], Real const & nx,
                    Real const & ny)
{
/*
  #Version  1.0. Released 1/1/99. Marco Manzini 1999

  #Purposes: Computed Gudonov numerical fluxes
   (This part describes what the routine does)

  #Input: lsol  -> left state   (density, ux, uy, pressure)
          rsol  -> right state  (,,)
          nx,ny -> references to the 
                   outward oriented components of normal of control volume
   (This part describes the input 
  
  #Output nflux -> numerical fluxes

  #Preconditions: lsol and rsol, should be valid states. 
                  nx*nx + ny*ny =1   

  #Preconditions Tests
                  Euler::ok.state should be used for the validity check of 
                  lstate and rstate.
                  
                  
  #Postconditions: nflux returns the numerical flux according to Gudunov
                   method. 

  #Postcondition Tests Not provided


*/
\end{verbatim}

In the documentation we provide a few keywords, identified by \verb!#!.
\begin{description}
\item The \emph{Version} keyword introduce a comment line which
  contains a reference number and \emph{DATE} which could be used to
  identify the revision of the routine we are using. This is
  particularly important for functions whose implementation may
  change. In particular, The keywords \emph{experimental}, and
  \emph{validated} should be respectively used to indicate software
  still on the experimental stage (no extensive validations made), and
  software already validated by tests. The name of the principal
  author should appear as well.
\item The \emph{Purposes} keyword introduces a brief description of
  the function scopes. Without establishing hard rules, the details
  contained in this part should increase the more the function usage
  is complex, the more it is near to the ``final user'' (in particular
  a detailed description should be done of user interface functions).
  In addition, it is expected that this part should be more carefully
  written in case of \emph{validated} software than
  \emph{experimental} one.
\item The \emph{Input} keyword introduces the description of the
  input arguments. A constant input argument is an argument which is
  \emph{never} modified by the function. Whenever a constant input is
  passed by reference it \emph{should be indicated by the C++ keyword
    \texttt{const}, i.e.\  it should be made a constant reference}. This
  help both the user and the compiler. Some routines may take
  additional input data from a file (or in general from an input
  stream). In that case, details should be indicated in this section.
\item The \emph{Output} keyword introduces a brief description of
  the function outputs, which may be  manifold:
  \begin{enumerate}
  \item A non-constant reference argument (note that a non-constant
    reference may be also an input, in that case its description will
    be repeated in the input section);
  \item A return value for the function;
  \item An output stream.
  \end{enumerate}
\item The \emph{Preconditions} keyword contains the conditions the
  input should satisfy in order to be proper. In particular the
  \emph{Preconditions Tests} section contains the lists of methods
  which should be used to verify if the preconditions are satisfied.
  The preconditions test \emph{should be automatically activated if
    the compiler switch \texttt{-DTEST\_PRE} has been used} (see
  section on debugging).
\item The \emph{Postconditions} keyword contains the conditions the
  output should satisfy in order to be proper.In particular the
  \emph{Postconditions Tests} section contains the lists of methods
  which should be used to verify if the postconditions are satisfied.
  The postconditions tests \emph{should be automatically activated if
    the compiler switch \texttt{-DTEST\_POS} has been used} (see
  section on debugging).
\item Array bounds checking is switched on by the \texttt{TEST\_BOUNDS} switch.
\end{description}
\subsection{Class documentation}
\begin{verbatim}
class FiniteElement : public GenericFiniteElement
{
/*
  #Version  1.0. Released. Giulio Cesare 89 A.C.

  #Purposes: This class contains an implementation of the SPQR finite
             elements.

  #Public data:
   
  #Private data:
           
  #Invariants                

  #Invariant Tests
                  

*/
\end{verbatim}
\begin{description}
\item The keyword \emph{version} has the same meaning as for the 
case of a function.
\item \emph{Purposes} is used to introduce a brief but exhaustive
description of tha class scopes.
\item \emph{Public Data} introduces a description of any public
data which the class contains, in particular static data.
\item The keyword \emph{Private data} should contain a description of
  the private data. It is used to give information to a potential
  programmer of the class. Thus it is required only for classes which
  are likely to be modified and for public data whose description if
  felt necessary for a clear understanding of the algorithms
  implemented in the classes method.
\item The keyword \emph{Invariants} tags the description of the
  class invariant quantities. These are properties that the class
  public and private data should satisfy if the class is in a correct
  state.  For instance, one may impose that each instance of the class
  \emph{Positive\_Definite\_Matrix}, used for positive definite
  matrices, should satisfy the condition that the minimum eigenvalue
  of the stored matrix is positive.
\item One may than devise a test, which will be a class method and
  will be described under the sub-item \emph{Invariants tests},
  which verify that condition. The test should be used for debugging
  purposes and activated when the compiler flag \texttt{-DTEST\_INV} is
  switched on (see section on debugging).
\end{description}

\subsection{File documentation}
Since class and methods are declared in a \texttt{.h} file, the first
line of the file will contain as well some of the documentation.
In particular the \emph{Version} and the  \emph{Purpose} field.
For the \emph{Version}  field, we will follow the convention that
\begin{itemize}
\item If a class does not contain it, the value of the file
containing the class is used;
\item If a function description does not contain it and the function
is a method, the one of the class applies, otherwise that of the file
containing the function declaration.
\end{itemize}

% 
\section{General Nomenclature and Programming Convention}
\subsection{Coding Standards}
We will follow the following simple rules:
\begin{itemize}
\item All declaration in \texttt{*.h} files. Method definitions (a part
  inlined methods or other possible special cases) are in a
  \texttt{cc} file with the same name;
\item No inlined methods definition within classes declarations, unless
they are \textbf{VERY} short. Inlined methods will be defined 
immediately after class declaration.
\item Use \texttt{INLINE} macro, instead of \texttt{inline}. The
  inline macro is switched off if \texttt{-DNOINLINE} is used.
\item Variables and classes must have names which recall their usage.
  Public variables named \texttt{a} or \texttt{pippo} are forbidden.
\item Class name have their first letter UPPER case, if the
    name is formed by many ``words'' the word initial letter is also
    upper case.  Ex.: \texttt{class MySimpleArray}.
\item \emph{Public} variables and functions names follow the same rule
as class names, a part from teh fact that the first letter is \emph{always} lower case. Example: \emph{Real pressure}, \texttt{Matrix \& computeMassMatrix(FiniteElement const \& localElement)}
\item Avoid function declarations without argument name. 
That is
\begin{verbatim}
Matrix & computeMassMatrix(FiniteElement const & localElement);
\end{verbatim}
should be used instead of
\begin{verbatim}
Matrix & computeMassMatrix(FiniteElement const & );
\end{verbatim}                                
This greatly helps understanding how the function works.
\item Use \texttt{const} keyword when possible. It helps the compiler
  and the human being reading the code. Moreover, it enhance
  debugging.
\item Typedefs aliases name follow the same rule as classes names.
\item Use the typedefs aliases \texttt{Real}, \texttt{Int} and
  \texttt{Uint}, instead of the in-built types \texttt{float},
  \texttt{Int} and \texttt{unsigned int}.  It helps making code
  changes afterwards. 
\item C preprocessor macros name are ALL UPPERCASE.
\item All principal classes will have a method
\begin{verbatim}
void  showMe(ostream & logStream);
\end{verbatim}
  which output on \texttt{logStream} (which may be the standard output
  or a file)  the state of the class (i.e.\  the content of the global
  and local variables.  Example (to be verified)
\begin{verbatim}
class AStupidArray {
public:
void addToArray(Real const & value);
...
void  showMe(ostream & logStream);
private  
_myArray[4][4];
}
void AStupidArray::showMe(ostream & logStream)
{
for(int i =0; i<4; ++i)
 {
   for(int j =0; j<4; ++i)
    logSteam << ''_myArray['' << i << '','' << j << ''] ''<<_myArray[i][j]<< endl;
 }
}
\end{verbatim}
\end{itemize}
We will also folly the coding practices outlined in the document
\textbf{Coding Standards} elaborated by \emph{The Laboratory for
  Scientific Computing, University of Notre Dame}, available
searching the Lab home page at \texttt{http:///www.lsc.nd.edu}.
                                
%
\section{Static and dynamic polymorphism}
%

% 
\section{Standard Template Library}
%
\subsection{Containers} We will make extensive use of STL containers
(in particular vectors and maps). This especially in the mesh handler.
\subsection{Strings}
%
We will use STL strings classes, much more flexible than C-style
strings.
\subsection{valarray}
%
\texttt{valarray} class is still not available in gnu c++. Then we
will avoid it for the moment. Yet, we will consider its adoption when correctly implemented in g++.
%
\section{Namespaces}
%
Still too early, but we suggest to read the C++ reference manual. They
may be of help to avoid name clashing in the future.
%
%
\section{Debugging and assertions} 
% 
\section{Debugging}
Debugging is essential! 
In the following we address the following issues
\begin{itemize}
\item Usage of \texttt{assert}s;
\item Usage of \texttt{cpp}  \texttt{-D} compiler switch for
switching on/off debugging at different level;
\item Usage of \texttt{cpp} \texttt{define}s to implement useful
macros and reduce typing (and typing errors!);
\item Debugging inlined functions.
\end{itemize}
\subsection{Assertions}
Assertion is a way of control the satisfaction of certain conditions.
If the condition is not satisfied an error message is printed and the
program is aborted.  

The C header file \texttt{assert.h} provide some basic utilities for
assertion, in particular the command \texttt{assert(b)}, will abort
the program if $b=0$ (i.e.\  false) giving an indication of the file and
the line where the assertion has failed. The \texttt{assert} is in
fact a \texttt{cpp} macro which can be turned off by setting the
compiler switch \texttt{-DNDEBUG} (NOT debug). It is rather crude,
thus we propose something similar but more sophisticated, following what
has been done by E.Berolazzi and M.Manzini in the \texttt{P2MESH} Library.
We define the following set of macros:

\begin{tabularx}{\textwidth}{|l|X|}
  \hline \texttt{ERROR\_MSG(A)} & Basic error message handler, used by
  all other macros.  It is NEVER switched off. It prints the stream
  \texttt{A} (which may be a string of something more complex) on the
  standard error \texttt{cerr} stream, lus an indication of the line and
  file where the condition occurred,
  and aborts the program.\\
  \texttt{ASSERT(X,A)} & Verifies condition \texttt{A} (which will be
  a logical expression), if not satisfied it will abort the program,
  writing the stream \texttt{A} plus an indication of the line and
  file where the condition occurred. It is switched \textbf{off} ONLY
  if
  \texttt{-DNODEBUG} is used at compile level. This should be used for the basic error checks that we would normally let active.\\
  \texttt{ASSERT\_PRE(X,A)} & Specialised \texttt{ASSERT} for
  \textit{preconditions} (see documentation section). It is switched
  \textbf{on} ONLY if \texttt{-DTEST\_PRE} of \texttt{-DTEST\_ALL} is
  used at compilation time.\\
 \texttt{ASSERT\_POS(X,A)} & Specialized \texttt{ASSERT} for
  \textit{postconditions} (see documentation section). It is switched
  \textbf{on} ONLY if \texttt{-DTEST\_POS} of \texttt{-DTEST\_ALL} is
  used at compilation time.\\
 \texttt{ASSERT\_INV(X,A)} & Specialized \texttt{ASSERT} for
  \textit{invariants} (see documentation section). It is switched
  \textbf{on} ONLY if \texttt{-DTEST\_INV} of \texttt{-DTEST\_ALL} is
  used at compilation time.\\\hline
\end{tabularx}

The last three asserts allow to modulate testing of pre, post
conditions and invariants. \texttt{ASSERT} macro have a \texttt{0}ed
version (e.g. \texttt{ASSERT0(X,A)}) which is never turned off.  An
example
\begin{verbatim}
bool complicatedCondition(const Matrix & A);
.....
.....
ASSERT(a>0,''The quantity a should be positive. Instead it is equal to ''<< a) ;
....
....
ASSERT_PRE(complicatedCondition(B),''Precondition test on matrix B not satisfied'') ;
...
...
\end{verbatim}
The \texttt{ASSERT}'s macros (c preprocessor macros) are defined in
the \texttt{lifeV.h} file. (YES, until somebody else give me another
name, I will call the software \textit{lifeV}, in memory of the all
\textit{life}s experienced in the past (I skipped III and IV, just for fun).

\subsection{Compiler Switches and define's}
\begin{itemize}
\item \texttt{-DTEST\_BOUND}  switches on bound checking in all
  arrays-like structure used in the library.
\item\texttt{-DTEST\_PRE} Preconditions are checked.
\item\texttt{-DTEST\_INV} Invariants are checked.
\item\texttt{-DTEST\_POS} Postconditions are checked.
\item \texttt{-DNOINLINE} switches of all \texttt{INLINE} macros.
  \texttt{INLINE} macros should be used in place of \texttt{inline}, in
  order to be able to switch off in-lining during debugging (in-lining
  makes debugging more complicated).
\item\texttt{-DTEST\_ALL} Equivalent to \texttt{-DTEST\_BOUND -DTEST\_PRE -DTEST\_INV
    -DTEST\_POS -DNOINLINE}
\item \texttt{-DSAVEMEMORY} Does not explicitely store some arrays, but rebuild the information
each time needed. It saves some memory at the expense of computional time.
\item  \texttt{INT\_BCNAME} Uses as name for BConditions just an integer and not
  a string. It saves memory at the expense of a more cryptic program.
\end{itemize}

\section{Type nemas, typedefs and template arguments}

It is very important to follows the convention for Type and Template
Arguments names. C++ allows to give alias to typenames with the
instruction \texttt{typedef} and this is a very useful practice, in
particular if \textbf{a common set of names is constantly used} which
correspond to a clearly identified set of classes. Before recalling the
main Types and setting up their general use we wish to recall some basic 
facts.

$\bullet$ Public typedefs are inherited.

Example
\begin{verbatim}

 class a {
public:
        typedef float Real;
 ....
};

class b: public a
{
....
};
\end{verbatim}

Now,  \verb!b::Real! is in fact a \texttt{float}!
   \medskip

$\bullet$ It is a good practice to expose with a typedef the main
  template arguments of a class template. 

Example: 

\begin{verbatim}
template<typename GEOSHAPE>
class AFiniteElement{

public:
        typedef  GEOSHAPE GeoShape;
........
};
\end{verbatim}

So that it os possible to do
\begin{verbatim}
typedef AFiniteElement<Tetra> ATetraElement;


int main()
{
        ATetraElement pippo;
        ATetraElement::GeoShape theShape;
}
\end{verbatim}
In this way the user does not have to remember the actual definition of
\texttt{ATetraElement} to access the typename of the Basic Geometric
Shape.

Remeber that a special care should be used when a templete class is
derived by a template argument:

\begin{verbatim}

class ABaseClass
public:
typedef Triangle GeoShape;
typedef std:complex<float> Complex;
...
};

template<typename BASE>
class Derived: public BASE
{
public:
  typedef BASE Base; // I expose the templete argument

// I need this since GeoShape is USED in the definition of this class

  typedef typename BASE::GeoShape  GeoShape; 
  Geoshape a; // Here I use GeoShape
};

int main()
{
 Derived<ABaseClass> pippo;
 Derived<ABaseClass>::Complex a; //OK
 Derived<ABaseClass>::GeoShape b; //OK
 .....
}
\end{verbatim}

I do not need to make visible the type \texttt{Complex} inside the
\texttt{Derived<BASE>} class templete since it is not used in the
definition (and I can rely to have it available by inheritance when I
instantiate the template class), I need instead to use the \texttt{typedef
  typename} declaration for \texttt{GeoShape} since it is used inside
the definition of \texttt{Derived<BASE>}.

SONO QUI

complicated by the fact that often the template argument name is the
same of the tepedef I would like to expose. For instance (an example
...) this WILL NOT WORK

template <typename GeoShape>
class FE{
typedef GeoShape Geoshape;
};

since the typedef will be ambiguous. One will need then change it into
something like

template <typename GEOSHAPE>
class FE{
        typedef GEOSHAPE Geoshape;
};

now it will work and inside the class template definition I can use
GEOSHAPE or GeoShape, more or less equivalently (even if, from a
logical point of view, they are different beasts).

We may decide to indicate all template arguments all upper case, yet
this will require too may changes... so I will just change the ones
which are needed to be changed. I prefer to change the name of the
template argument while having the name of the typedef conforming to the general
rules. After all the template argument name is a dummy, while it is
important to be consistent with the type names!.

To make myself clearer (I hope!) a GeoShape (which is the typename for a
generic Basis Geometric Shape: LinearTriangle for example) will alway
typedef eather as GeoShape (generic name) or FaceShape, EdgeShape etc
(specific names) and NOT as My\_Geoshape or GeoShapeType or something
like that. However, I will momentarily keep (for consistency reason)
also these 'old' typedefs, so to avoid too many incompatibilities. Yet
they should gradually go away.

d) There is a clear difference of style between my programming and that
of Jean Fred (who is more experience in C++ than I, after all). I use a
lot inheritance as a mechanism of building more complex classes from
simpler ones. He prefers typedef containement.

I.E.

I would have written (is just an example, it does not correspond to an
actual class of the library!!!)

template<BASISFCT,GEOMAP,Quadrule>
class Fe: public BASISFCT, public GEOMAP
{
public:
typedef BASISFCF BasisFct;
typedef GEOMAP GeoMap;

... etc etc ....
};

while jean Fred prefers

template<BASISFCT,GEOMAP,Quadrule>
class Fe
{
public:
typedef BASISFCF BasisFct;
typedef GEOMAP GeoMap;

... etc etc ....
};

and then access BasisFct and GeoMap interfaces through the typedef.  In
the first case instead, the typedef statement is used only to expose the
template arguments, while the interfaces of BasisFct and GeoMap are
directly available through inheritance.

We may argue for months to what is best and when it is best to use one
strategy or the other... In any case, the existance of both styles in
the library MAKES EVEN MORE IMPORTANT to have a coherent strategy for
type names and typedefs!!!!!
