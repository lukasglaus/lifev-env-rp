
\chapter{Generalities}
\label{cha:generalities}

\section{Scope of the document}
\label{sec:scope-document}

This is a ``working document'' meant to drive the development of the
software library \thelibrary. It is an informal document. All the
people involved in the library development (from now on called the
\emph{authors}) are supposed to freely contribute to its writing.

The major objectives of this document is manifold:
\begin{enumerate}
\item Set up a set of rules regarding program development, coding rules
  and standards, which must be followed by the software developers;
\item Provide the architectural design for the libraries and classes
  to be developed, with a special stress on public interfaces;
\item Provide a subtask subdivision for the work to be done and a
  (possibly accurate) \emph{scheduling}.
\end{enumerate}

\section{Language and nomenclature convection}
\label{sec:lang-nomencl-conv}

%Because of many practical reasons, the document will be written in
%English, or at least the best English the authors can manage.

We will use \texttt{typesetting font style} to indicate parts of
actual computer code or name of variables, types, etc..
\textbf{Boldface} is used to mark portion containing rules which
should be followed during program development and \emph{emphasised}
text to indicate important concepts and nomenclature.

\section{Software Management}
\label{sec:software-management}

%This software is the result of the work of many people working in a
%coordinated fashion, some rules for software management
%must be set and agreed upon. A related problem is how to allow for the most ample discussion
%between the authors and, at the same time, coordinate  software
%production.


The software source, its documentation and all related documents (this
one included) is kept in a repository under revision control
using \ixs{CVS}{versioning}\footnote{CVS stands for Concurrent Version System}.


A web site {\it\`a la} Sourceforge\footnote{\url{http://www.sourceforge.net}} \url{http://cmcsforge.epfl.ch} has been set up to host the source code and help the software management.
It requires that you open an account\footnote{\url{https://cmcsforge.epfl.ch/account/register.php}} there and that you declare your intentions to work on \lifev through an email to the development mailing list of \lifev.
Once you are part of the project, you have access to all the facilities: tracker, task manager, CVS repository, forums, document manager and a few other tools which are very useful if not absolutely essential to such a project.

CVS keywords like \verb!Id! and \verb!Log! should not be included in source files,
they cause many unnecessary conflicts at update/commit time. Use
\verb!cvs log! to get the information given by \verb!Id! or \verb!Log!.

\section{Compiling LifeV}
\label{compile-lifev}

In order to compile \lifev, certain requirements must be met in terms
of compilation tools, and libraries to link to.

\noindent In this document, we shall use \verb+$(top_srcdir)+ (aka \verb+$LIFEV_HOME+) to designate the top level \emph{source} directory
and \verb+$(top_builddir)+ to designate the top level \emph{build}
directory where \lifev has been \emph{configured}(see
sections~\ref{sec:compile-cvs} and~\ref{sec:comp-from-offic}) for a
compilation environment.  \verb+$(top_builddir)+ may be different from
\verb+$(top_srcdir)+, see for example the configure hint in
section~\ref{hint:configure}.

\subsection{Requirements}

\subsubsection{Compilation Environment}
\label{sec:comp-envir}

\lifev depends on a number of tools at compilation time, they are part
of the \ix{autotools} from the GNU
project\footnote{\url{http://www.gnu.org}}.

\begin{itemize}
\item \verb!libtool 1.5!\ixns{libtool}{autotools}
\item \verb!automake 1.7!\ixns{automake}{autotools}
\item \verb!autoconf 2.5x x>=2!\ixns{autoconf}{autotools}
\item \verb!g++-4.0 x>=0!\ixns{g++}{compilers}
\item \verb!a version of \verb mpi!
\item \verb!Trilinos!\ixns{aztec}{algebra} (which depends on \verb!mpi!, \verb!blas! and \verb!lapack!)
\item \verb!libumfpack4 (set of routines for solving unsymmetric sparse linear systems)
\end{itemize}

%\subsubsection{Runtime Environment}
%\label{sec:runtime-env}

%\begin{itemize}
%\item \verb!aztec!
%\end{itemize}

\subsection{Compilation from CVS}
\label{sec:compile-cvs}
You need first to have an account on \url{http://cmcsforge.epfl.ch} and
be part of the \lifev project, see~\ref{sec:software-management}.

\noindent First, you need to checkout \lifev. \ixv{CVS} has
been configured to use \ixv{ssh} and your \verb!ssh! keys to
access the

\begin{verbatim}
export CVS_RSH=ssh
cvs -z3 -d:ext:developername@cmcsforge.epfl.ch:/cvsroot/lifev-parallel co lifev-parallel
\end{verbatim}

then place yourself in the new directory :

\begin{verbatim}
cd lifev-parallel
\end{verbatim}

\noindent Second, you have to generate the compilation environment by typing:
\begin{verbatim}
make -f Makefile.cvs
\end{verbatim}

\noindent Third, you have to execute the script
\ixvs{configure}{autotools}
it will automatically check the availability of the needed components
for \lifev compilation. You type:

\begin{verbatim}
configure --with-trilinos-lib=<your trilinos libs path> --with-trilinos-include=<your trilinos include path> --enable-opt --with-openmpi --prefix=<your lifev library install path>

\end{verbatim}

\noindent Finally, you just have to use \ixv{make} to compile \lifev libraries. You type
\begin{verbatim}
make
make install
\end{verbatim}
Optionally, you use \verb!-j x(x=2,3 or 4)! to speed up the compilation time.

\begin{hint}{Configure}
  \label{hint:configure}
  \verb!configure! is extremely powerful and allows you to maintain
  several concurrent build directories. Typically during development
  and testing, you need to have \lifev compiled with debugging flags,
  optimization flags and/or profiling flags. However combining
  optimization and debugging flags does not necessarily produce
  anything useful for debugging and testing purposes.

  With configure it is possible to compile a code in a directory which
  is different from the source directory. This is extremely useful to
  tacke our problem. Let's consider given the source directory to be
  store in the environment variable \verb!$LIFEV_HOME! which is not
  mandatory.

Here is what you can do to compile with debugging flags:
\begin{verbatim}
cd $LIFEV_HOME
mkdir debug
cd debug
CXXFLAGS="-g3 -O0" ../configure
make
\end{verbatim}
Here we have \verb+$(top_builddir) == $LIFEV_HOME/debug+.

Here is what you can do to compile with optimization flags:
\begin{verbatim}
cd $LIFEV_HOME
mkdir optimized
cd optimized
CXXFLAGS="-O2" ../configure
make
\end{verbatim}
Here we have \verb+$(top_builddir) == $LIFEV_HOME/optimized+.

\noindent Note that you may have several concurrent \verb+$(top_builddir)+

\noindent Note that \verb!configure! will fail if you have already compiled
\lifev in the source directory.


\end{hint}


\subsection{Compilation from Official Distribution}
\label{sec:comp-from-offic}
The \lifev project provides releases, they are named using the following convention:
\begin{center}
\verb!lifev-x.y.z.tar.gz!
\end{center}

Here is what you have to do:

\begin{enumerate}
\item download \lifev release \verb!lifev-x.y.z.tar.gz!
\item unpack it
\begin{verbatim}
tar xzf lifev-x.y.z.tar.gz
\end{verbatim}
\item configure it
\begin{verbatim}
cd lifev-x.y.z
configure
\end{verbatim}
\item compile it
\begin{verbatim}
make
\end{verbatim}
\end{enumerate}

Some comments in section\ref{sec:compile-cvs} apply also here.

\subsection{Compiling Testsuite}

\noindent \lifev comes with a testsuite covering a lot of features. It is located in the directory \verb+testsuite+
\begin{verbatim}
|-- data
|-- test_bdf
|-- test_darcy
|-- test_essentialbc
|-- test_fe
|-- test_fsi_newton
|-- test_fsi_picard
|-- test_linearelasticity
|-- test_matrix
|-- test_mesh
|-- test_robin
|-- test_naturalbc
|-- test_ns_bdf
|-- test_ns_cyl
|-- test_ns_sstress
|-- test_p2
|-- test_postproc
`-- test_q1
\end{verbatim}

\noindent In order to compile the testsuite, you need the following steps
\begin{verbatim}
cd $(top_builddir)/testsuite
make check
\end{verbatim}

\noindent If you just want to compile a specific test, say \verb+test_darcy+:
\begin{verbatim}
cd $(top_builddir)/testsuite/data
make check
cd $(top_builddir)/testsuite/test_darcy
make test_darcy _OR_ make check
\end{verbatim}

\noindent Since most tests are using meshes that are located in
\verb+$(top_srcdir)/testsuite/data+, it is mandatory to execute \verb+make check+ in
\verb+$(builddir)/testsuite/data+ in order to create the proper
symlinks to the meshes when \verb+$(top_builddir) != $(top_srcdir)+.

%
%%%%%%%%%%%%% Some Settings for emacs and auc-TeX
% Local Variables:
% TeX-master: "lifev-dev"
% TeX-command-default: "PDFLaTeX"
% TeX-parse-self: t
% TeX-auto-save: t
% TeX-auto-regexp-list: TeX-auto-full-regexp-list
% eval: (ispell-change-dictionary "american")
% End:
%
